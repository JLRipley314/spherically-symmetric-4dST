\documentclass[a4paper,11pt]{article}
\pdfoutput=1 % if your are submitting a pdflatex (i.e. if you have
             % images in pdf, png or jpg format)

\usepackage{jheppub} % for details on the use of the package, please
                     % see the JHEP-author-manual

\usepackage[T1]{fontenc} % if needed



\title{\boldmath Notes: Spherically symmetric black hole dynamics in $4\partial ST$ gravity}


%% %simple case: 2 authors, same institution
%% \author{A. Uthor}
%% \author{and A. Nother Author}
%% \affiliation{Institution,\\Address, Country}

% more complex case: 4 authors, 3 institutions, 2 footnotes
\author[a,1]{Justin L. Ripley}

% The "\note" macro will give a warning: "Ignoring empty anchor..."
% you can safely ignore it.

\affiliation[a]{
   DAMTP, Centre for Mathematical Sciences, University of Cambridge,  
   \\ Wilberforce Road, Cambridge CB3 0WA, UK.
}

% e-mail addresses: one for each author, in the same order as the authors
\emailAdd{lloydripley@gmail.com}
%\emailAdd{weast@perimeterinstitute.ca}
%\emailAdd{Thomas.Sotiriou@nottingham.ac.uk}

\abstract{
   Notes on the equations of motion and numerical implementation.
   If you end up using this code, while some implementation details
   have changed since original publication please consider
   citing \cite{Ripley:2020vpk}.
   }



\begin{document} 
\maketitle
\flushbottom

\section{Introduction}
\label{sec:intro}

We set $G=c=1$, our metric signature is $-+++$,
we use lower case Latin indices to index spacetime dimensions,
the Riemann tensor is
$R^a{}_{mbn}=\partial_{b}\Gamma^{a}_{mn}-\cdots$. 


%\begin{figure}[tbp]
%\centering % \begin{center}/\end{center} takes some additional vertical space
%\includegraphics[width=.45\textwidth,trim=0 380 0 200,clip]{img1.pdf}
%\hfill
%\includegraphics[width=.45\textwidth,origin=c,angle=180]{img2.pdf}
%% "\includegraphics" is very powerful; the graphicx package is already loaded
%\caption{\label{fig:i} Always give a caption.}
%\end{figure}

%=============================================================================
\section{Equations of motion}
   We consider the theory (``$4\partial ST$ gravity'', e.g. \cite{Kovacs:2020ywu})
\begin{align}
   S
   =
   \frac{1}{8\pi}\int d^4x\sqrt{-g}\left(
      \frac{1}{2}R
   +  X
   -  V(\phi)
   +  \alpha(\phi)X^2
   +  \beta(\phi)\mathcal{G}
   \right)
   ,
\end{align}
   where
\begin{subequations}
\begin{align}
   X
   \equiv&
-  \frac{1}{2}\left(\nabla\phi\right)^2
   ,\\
   \mathcal{G}
   \equiv&
   \frac{1}{4}\delta^{abcd}_{efgh}
   R^{ef}{}_{ab}R^{gh}{}_{cd}
   ,\\
   \delta^{abcd}_{efgh}
   \equiv&
   4!\delta^a_{[e}\delta^b_f\delta^c_g\delta^d_{h]}
   .
\end{align}
\end{subequations}
   Varying the action with respect to the scalar field and metric field
gives us the equations of motion:
\begin{align}
\label{eq:eom_edgb_scalar}
   E^{(\phi)}
   &\equiv
      \Box\phi
   -  V^{\prime}\left(\phi\right)
   \nonumber\\&
   +  2\alpha\left(\phi\right)X \Box\phi
   -  2\alpha\left(\phi\right)\nabla^a\phi\nabla^b\phi\nabla_a\nabla_b\phi
   \nonumber\\&
   -  3\alpha^{\prime}\left(\phi\right)X^2
   +  \beta^{\prime}\left(\phi\right)\mathcal{G}
   =
   0
   ,\\
\label{eq:eom_edgb_tensor}
   E^{(g)}_{ab}
   &\equiv
   R_{ab}
-  \frac{1}{2}g_{ab}R
   \nonumber\\&
-  \nabla_a\phi\nabla_b\phi
+  \left(-X+V\left(\phi\right)\right)g_{ab}
   \nonumber\\&
-  2\alpha\left(\phi\right)X\nabla_a\phi\nabla_b\phi
-  \alpha\left(\phi\right)X^2g_{ab}
   \nonumber\\&
+  2\delta^{efcd}_{ijg(a}g_{b)d}R^{ij}{}_{ef}
   \nabla^g\nabla_c\beta\left(\phi\right) 
   =
   0
   .
\end{align}
%=============================================================================
\section{Numerical implementation}
%=============================================================================
\subsection{Coordinates}
   Our procedure for numerically evolving this system in spherical symmetry
will largely follow the analysis of \cite{Ripley:2020vpk}.
   We will numerically evolve this system in Painlev\'{e}-Gullstrand (PG)-like
coordinates:
\begin{align}
   ds^2
   =
-  N(t,r)dt^2
+  \left(dr+N(t,r)S(t,r)dt\right)^2
+  r^2\left(d\vartheta^2+\mathrm{sin}^2\vartheta d\varphi^2\right)
   ,
\end{align}
   which are so-named as in the static limit black hole solutions are given
by the Painlev\'{e}-Gullstand coordinate representation of the Schwarzschild
black hole: $N=1$, $S=\sqrt{2m/r}$.

   We define the variables:
\begin{subequations}
\begin{align}
   Q
   \equiv&
   \partial_t\phi
   ,\\
   P
   \equiv&
   \frac{1}{N}\partial_t\phi-S Q
   ,
\end{align}
\end{subequations}
   and obtain the following system of evolution equations for
$\phi$, $Q$, and $P$:
\begin{subequations}
\begin{align}
   E_{(\phi)}
   \equiv&
   \partial_t\phi
-  N\left(P+S Q\right)
   = 0
   ,\\
   E_{(Q)}
   \equiv&
   \partial_tQ
-  \partial_r\left(N\left[P+S Q\right]\right)
   = 0
   ,\\
   E_{(P)}
   \equiv&
   \mathcal{A}_{(P)}\partial_tP
+  \mathcal{F}_{(P)}
   = 0
   ,
\end{align}
\end{subequations}
   where $\mathcal{A}_{(P)}$ and $\mathcal{F}_{(P)}$
are functions of $P,Q,\phi,N,S$, and their radial derivatives. 
We can solve for the metric field variables using the
Hamiltonian ($\mathcal{H}=0$) and momentum $\mathcal{M}_r=0$)
constraint equations.
Using the unit timelike vector to our $t=const.$ hypersurfaces:
$n_a=(-N,0,0,0)$ these are defined to be 
\begin{align}
\label{eq:metric_field_odes}
   \mathcal{H}
   \equiv
   E^{(g)}_{ab}n^an^b
   \propto&
   \left(
      1
   -  8\beta^{\prime}\frac{Q}{r}
   -  12\beta^{\prime}\frac{P S}{r}
   \right)
   \partial_rS
+  \left(
      S
   -  9\beta^{\prime}\frac{QS}{r}
   -  12\beta^{\prime}\frac{PS^2}{r}
   \right)
   \frac{\partial_rN}{N}
   \nonumber\\&
-  4\beta^{\prime}\frac{S}{r}\partial_rQ  
-  4\beta^{\prime\prime}\frac{Q^2S}{r}
+  \frac{S}{2r}
-  \frac{1}{2}\frac{r\rho}{S}
   ,\\
   \mathcal{M}_r
   \equiv
   E^{(g)}_{ar}n^a
   \propto&
   \left(
      1
   -  8\beta^{\prime}\frac{Q}{r}
   -  8\beta^{\prime}\frac{PS}{r}
   +  4\beta^{\prime}\frac{QS^2}{r}
   \right)
   \frac{\partial_rN}{N}
+  4\beta^{\prime}\frac{QS}{r}\partial_rS
+  4\beta^{\prime}\frac{S}{r}\partial_rP
   \nonumber\\&
+  4\beta^{\prime\prime}\frac{P Q S}{r}
-  \frac{1}{2}\frac{r j_r}{S}
   ,
\end{align}
   where $r$ is the radial index, and
\begin{subequations}
\begin{align}
   \rho
   \equiv&
   n^an^bT_{ab}
   ,\\
   j_r
   \equiv&
   n^aT_{ar}
   ,\\
   T_{ab}
   \equiv&
   \left(
      1+ 2 \alpha X
   \right)
   \nabla_a\phi\nabla_b\phi
+  g_{ab}
   \left(
      X
   -  V 
   +  \alpha X^2
   \right)
   .
\end{align}
\end{subequations}
   For our initial data, we freely specify $\phi$ and $P$.
We obtain $Q=\partial_r\phi$, and solve for the metric fields using
the Hamiltonian and momentum constraint equations.
There is a residual gauge symmetry which allows us
to rescale $N$ by a constant value at each time step;
we use this freedom to rescale $N$ such
that $\lim_{r\to\infty}N=1$.
A trapped surface is signaled by $S<1$;
thus to begin with black hole initial data we set $S<1$ at some
excision point.

   As independent residual checks on our solutions, we plug our
solution into the $rr$ component of the tensor equations, $E^{(g)}_{rr}=0$,
and consider $Q-\partial_r\phi=0$, both of which should converge
to zero with higher resolution.

   We compactify the radial coordinate in the code:
\begin{align}
   x
   \equiv
   \frac{r}{1+r/L}
   ,
\end{align}
   where $L$ is the compactification radius.
%=============================================================================
\subsection{Form of scalar field potentials}
   We consider potentials of the form
\begin{subequations}
\begin{align}
   V\left(\phi\right)
   =&
   V_1\phi
+  \frac{1}{2}V_2\phi^2
+  \frac{1}{6}V_3\phi^3
+  \frac{1}{24}V_4\phi^4
   ,\\
   \alpha\left(\phi\right)
   =&
   \alpha_0
+  \alpha_1\phi
+  \frac{1}{2}\alpha_2\phi^2
+  \frac{1}{6}\alpha_3\phi^3
+  \frac{1}{24}\alpha_4\phi^4
   ,\\
   \beta\left(\phi\right)
   =&
   \beta_1\phi
+  \frac{1}{2}\beta_2\phi^2
+  \frac{1}{6}\beta_3\phi^3
+  \frac{1}{24}\beta_4\phi^4
   .
\end{align}
\end{subequations}
   Where all the subscripted quantities, e.g. $\alpha_2$, are constants
determined by the user in the {\bf setup\_run.py} file.
We do not put in a $V_0$ (a cosmological constant) as generically
spacetimes with cosmological constants are not aysmptotically flat,
and we do not put in a $\beta_0$ term as the Gauss-Bonnet scalar
does not constribute to the equations of motion in that case.
%==============================================================================
%\acknowledgments

%This is the most common positions for acknowledgments. A macro is
%available to maintain the same layout and spelling of the heading.

%\paragraph{Note added.} This is also a good position for notes added
%after the paper has been written.


\bibliographystyle{JHEP}
\bibliography{spherical}
\end{document}
